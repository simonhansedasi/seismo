\documentclass{article}
\usepackage{float}
\usepackage{graphicx}
\usepackage{amsmath}
\usepackage{listings}
\usepackage{color}
\usepackage{multicol}
\usepackage{cancel}
\definecolor{cadmiumgreen}{rgb}{0.0, 0.42, 0.24}
\lstset{frame=tb,
  language=R,
  aboveskip=3mm,
  belowskip=3mm,
  showstringspaces=false,
  columns=flexible,
  basicstyle={\small\ttfamily},
  numbers=none,
  numberstyle=\tiny\color{gray},
  keywordstyle=\color{blue},
  commentstyle=\color{dkgreen},
  stringstyle=\color{cadmiumgreen},
  breaklines=true,
  breakatwhitespace=true,
  tabsize=3
}
\usepackage[margin=0.75in]{geometry}
\setlength\parindent{0pt}

\title{Seismo 512 HW 3}
\date{1/20/2023}
\author{Simon-Hans Edasi}

\begin{document}

	\maketitle

(2 points) From Equations (2.34) and (2.35)  (P wave speed and Poisson ratio) derive expressions for the Lamé parameters in terms of the seismic velocities and density.




\begin{align*}
\alpha & =  \sqrt\frac{\lambda + 2 \mu}{\rho}   &   \beta & =    \sqrt\frac{\mu}{\rho}
\end{align*}
Write $\mu$ in terms of $\beta$, substitute in to the $\alpha$ expression and solve for $\lambda$.
\[
\mu = \rho\beta^{2} \quad \rightarrow \quad \alpha^{2} = \frac{\lambda + 2\left(\beta^{2}\rho\right)}{\rho} \quad \rightarrow \quad \lambda = \rho\left(\alpha^{2} - 2\beta^{2}\right)
\]
(3 points) Seismic observations of S velocity can be directly related to the shear modulus µ. However, P velocity is a function of both the shear and bulk moduli. For this reason, sometimes seismologists will compute the bulk sound speed, defined as
\[
V_c = \sqrt\frac{\kappa}{\rho}
\]
which isolates the sensitivity to the bulk modulus $\kappa$.\\

Derive an equation for Vc in terms of the P velocity, $\alpha$, and the S velocity. For the specific case of a Poisson solid, express Vc as a fraction of the P
Velocity.
\[
\kappa = \lambda + \left(\frac{2}{3}\right)\mu
\]
Using definitions from the previous answer for $\mu$ and $\lambda$, we can write.
\[
V_c = \sqrt{
\frac{\bcancel{ \rho } \left( \alpha^{2} - 2\beta^{2} \right) + \left( \frac{2}{3} \right)  \bcancel{ \rho }\beta^{2}  }
{\bcancel{ \rho }
}
}
= \sqrt{ \alpha^{2} - \frac{4}{3}\beta^{2}}
\]\\
For a Poisson solid: $\lambda = \mu$ and $\frac{\alpha}{\beta} = \sqrt{3}  \rightarrow \beta = \frac{\alpha}{\sqrt{3}}$ and we derived $\kappa = \lambda + \left(\frac{2}{3}\right)\mu = \frac{5}{3}\lambda$\\

\[
V_c = \sqrt\frac{\frac{5}{3} \lambda}{\rho} = \sqrt \frac{\frac{5}{3} \bcancel{\rho}  \left( \alpha^{2} - 2\beta^{2}  \right)}{\bcancel{\rho}} = \sqrt{ \frac{5}{3} \left( \alpha^{2} - 2\left( \frac{\alpha}{\sqrt{3}} \right)^{2} \right) }  = \sqrt{ \frac{5}{3} \alpha^2 - \frac{10}{9}\alpha^2} = \frac{\sqrt{5}}{3}\alpha
\]\\





(3 points) What is the P /S velocity ratio for a rock with a Poisson’s ratio of 0.30?\\
\[s
\sigma = \frac{ \lambda } { 2 \left( \lambda + \mu \right) } = 0.03 = \frac { \bcancel{\rho} \left( \alpha^{2} - 2\beta^{2} \right)} { 2 \left( \bcancel{\rho} \left(  \alpha^{2} - 2\beta^{2} \right) + \bcancel{\rho}\beta^{2} \right)} = \frac{ \alpha^{2} - 2\beta^{2} } { 2 \left( \alpha^{2} - \beta^{2} \right) }
\]
\[
0.6 = \frac{ \alpha^{2} - 2\beta{2} } {\alpha{2} - \beta^{2} } \quad \rightarrow \quad 0.6\alpha^{2} - 0.6\beta^{2} = \alpha^{2} - 2\beta^{2} \quad \rightarrow \quad 1.4\beta = 0.6\alpha \quad \rightarrow \quad \frac{ \alpha } { \beta } = 2
\]\\
(4 points) A sample of granite in the laboratory is observed to have a P velocity of 5.5 km/s and a density of 2.6 Mg/m3. Assuming it is a Poisson solid, obtain values for the Lam´e parameters, Young’s modulus, and the bulk modulus. Express your answers in pascals.\\

We know velocity $V_c = \sqrt{ \frac{ \frac{5}{3} \lambda }{\rho} } $

\[
\lambda = \frac{3}{5} V_c^{2} \rho = \frac{ \left( 30.25 \frac{m}{s} \right)^{2}}{7.8 * 10^{3} \frac{kg}{m^{3}} } = 3.88 * 10^{3} Pa
\] 


(4 points) Figure 2.6 shows surface displacement rates as a function of distance from the San Andreas Fault in California.\\
 

Consider this as a 2-D problem with the x-axis perpendicular to the fault and the y-axis parallel to the fault. From these data, estimate the yearly strain $\epsilon$ and rotation $\Omega$ tensors for a point on the fault. Express your answers as 2 × 2 matrices.\\

Assuming the crustal shear modulus is 27 GPa, compute the yearly change in the stress tensor. Express your answer as a 2 × 2 matrix with appropriate units.\\

If the crustal shear modulus is 27 GPa, what is the shear stress across the fault after 200 years, assuming zero initial shear stress?\\

[open-ended question] If large earthquakes occur every 200 years and release all of the distributed strain by movement along the fault, what, if anything, can be inferred about the absolute level of shear stress?\\

[open-ended question]What, if anything, can be learned about the fault from the observation that most of the deformation occurs within a zone less than 50 km wide? 





For example, see \eqref{myeq}.






\end{document}
