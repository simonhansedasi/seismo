\documentclass[11pt]{article}

    \usepackage[breakable]{tcolorbox}
    \usepackage{parskip} % Stop auto-indenting (to mimic markdown behaviour)
    

    % Basic figure setup, for now with no caption control since it's done
    % automatically by Pandoc (which extracts ![](path) syntax from Markdown).
    \usepackage{graphicx}
    % Maintain compatibility with old templates. Remove in nbconvert 6.0
    \let\Oldincludegraphics\includegraphics
    % Ensure that by default, figures have no caption (until we provide a
    % proper Figure object with a Caption API and a way to capture that
    % in the conversion process - todo).
    \usepackage{caption}
    \DeclareCaptionFormat{nocaption}{}
    \captionsetup{format=nocaption,aboveskip=0pt,belowskip=0pt}

    \usepackage{float}
    \floatplacement{figure}{H} % forces figures to be placed at the correct location
    \usepackage{xcolor} % Allow colors to be defined
    \usepackage{enumerate} % Needed for markdown enumerations to work
    \usepackage{geometry} % Used to adjust the document margins
    \usepackage{amsmath} % Equations
    \usepackage{amssymb} % Equations
    \usepackage{textcomp} % defines textquotesingle
    % Hack from http://tex.stackexchange.com/a/47451/13684:
    \AtBeginDocument{%
        \def\PYZsq{\textquotesingle}% Upright quotes in Pygmentized code
    }
    \usepackage{upquote} % Upright quotes for verbatim code
    \usepackage{eurosym} % defines \euro

    \usepackage{iftex}
    \ifPDFTeX
        \usepackage[T1]{fontenc}
        \IfFileExists{alphabeta.sty}{
              \usepackage{alphabeta}
          }{
              \usepackage[mathletters]{ucs}
              \usepackage[utf8x]{inputenc}
          }
    \else
        \usepackage{fontspec}
        \usepackage{unicode-math}
    \fi

    \usepackage{fancyvrb} % verbatim replacement that allows latex
    \usepackage{grffile} % extends the file name processing of package graphics
                         % to support a larger range
    \makeatletter % fix for old versions of grffile with XeLaTeX
    \@ifpackagelater{grffile}{2019/11/01}
    {
      % Do nothing on new versions
    }
    {
      \def\Gread@@xetex#1{%
        \IfFileExists{"\Gin@base".bb}%
        {\Gread@eps{\Gin@base.bb}}%
        {\Gread@@xetex@aux#1}%
      }
    }
    \makeatother
    \usepackage[Export]{adjustbox} % Used to constrain images to a maximum size
    \adjustboxset{max size={0.9\linewidth}{0.9\paperheight}}

    % The hyperref package gives us a pdf with properly built
    % internal navigation ('pdf bookmarks' for the table of contents,
    % internal cross-reference links, web links for URLs, etc.)
    \usepackage{hyperref}
    % The default LaTeX title has an obnoxious amount of whitespace. By default,
    % titling removes some of it. It also provides customization options.
    \usepackage{titling}
    \usepackage{longtable} % longtable support required by pandoc >1.10
    \usepackage{booktabs}  % table support for pandoc > 1.12.2
    \usepackage{array}     % table support for pandoc >= 2.11.3
    \usepackage{calc}      % table minipage width calculation for pandoc >= 2.11.1
    \usepackage[inline]{enumitem} % IRkernel/repr support (it uses the enumerate* environment)
    \usepackage[normalem]{ulem} % ulem is needed to support strikethroughs (\sout)
                                % normalem makes italics be italics, not underlines
    \usepackage{mathrsfs}
    

    
    % Colors for the hyperref package
    \definecolor{urlcolor}{rgb}{0,.145,.698}
    \definecolor{linkcolor}{rgb}{.71,0.21,0.01}
    \definecolor{citecolor}{rgb}{.12,.54,.11}

    % ANSI colors
    \definecolor{ansi-black}{HTML}{3E424D}
    \definecolor{ansi-black-intense}{HTML}{282C36}
    \definecolor{ansi-red}{HTML}{E75C58}
    \definecolor{ansi-red-intense}{HTML}{B22B31}
    \definecolor{ansi-green}{HTML}{00A250}
    \definecolor{ansi-green-intense}{HTML}{007427}
    \definecolor{ansi-yellow}{HTML}{DDB62B}
    \definecolor{ansi-yellow-intense}{HTML}{B27D12}
    \definecolor{ansi-blue}{HTML}{208FFB}
    \definecolor{ansi-blue-intense}{HTML}{0065CA}
    \definecolor{ansi-magenta}{HTML}{D160C4}
    \definecolor{ansi-magenta-intense}{HTML}{A03196}
    \definecolor{ansi-cyan}{HTML}{60C6C8}
    \definecolor{ansi-cyan-intense}{HTML}{258F8F}
    \definecolor{ansi-white}{HTML}{C5C1B4}
    \definecolor{ansi-white-intense}{HTML}{A1A6B2}
    \definecolor{ansi-default-inverse-fg}{HTML}{FFFFFF}
    \definecolor{ansi-default-inverse-bg}{HTML}{000000}

    % common color for the border for error outputs.
    \definecolor{outerrorbackground}{HTML}{FFDFDF}

    % commands and environments needed by pandoc snippets
    % extracted from the output of `pandoc -s`
    \providecommand{\tightlist}{%
      \setlength{\itemsep}{0pt}\setlength{\parskip}{0pt}}
    \DefineVerbatimEnvironment{Highlighting}{Verbatim}{commandchars=\\\{\}}
    % Add ',fontsize=\small' for more characters per line
    \newenvironment{Shaded}{}{}
    \newcommand{\KeywordTok}[1]{\textcolor[rgb]{0.00,0.44,0.13}{\textbf{{#1}}}}
    \newcommand{\DataTypeTok}[1]{\textcolor[rgb]{0.56,0.13,0.00}{{#1}}}
    \newcommand{\DecValTok}[1]{\textcolor[rgb]{0.25,0.63,0.44}{{#1}}}
    \newcommand{\BaseNTok}[1]{\textcolor[rgb]{0.25,0.63,0.44}{{#1}}}
    \newcommand{\FloatTok}[1]{\textcolor[rgb]{0.25,0.63,0.44}{{#1}}}
    \newcommand{\CharTok}[1]{\textcolor[rgb]{0.25,0.44,0.63}{{#1}}}
    \newcommand{\StringTok}[1]{\textcolor[rgb]{0.25,0.44,0.63}{{#1}}}
    \newcommand{\CommentTok}[1]{\textcolor[rgb]{0.38,0.63,0.69}{\textit{{#1}}}}
    \newcommand{\OtherTok}[1]{\textcolor[rgb]{0.00,0.44,0.13}{{#1}}}
    \newcommand{\AlertTok}[1]{\textcolor[rgb]{1.00,0.00,0.00}{\textbf{{#1}}}}
    \newcommand{\FunctionTok}[1]{\textcolor[rgb]{0.02,0.16,0.49}{{#1}}}
    \newcommand{\RegionMarkerTok}[1]{{#1}}
    \newcommand{\ErrorTok}[1]{\textcolor[rgb]{1.00,0.00,0.00}{\textbf{{#1}}}}
    \newcommand{\NormalTok}[1]{{#1}}

    % Additional commands for more recent versions of Pandoc
    \newcommand{\ConstantTok}[1]{\textcolor[rgb]{0.53,0.00,0.00}{{#1}}}
    \newcommand{\SpecialCharTok}[1]{\textcolor[rgb]{0.25,0.44,0.63}{{#1}}}
    \newcommand{\VerbatimStringTok}[1]{\textcolor[rgb]{0.25,0.44,0.63}{{#1}}}
    \newcommand{\SpecialStringTok}[1]{\textcolor[rgb]{0.73,0.40,0.53}{{#1}}}
    \newcommand{\ImportTok}[1]{{#1}}
    \newcommand{\DocumentationTok}[1]{\textcolor[rgb]{0.73,0.13,0.13}{\textit{{#1}}}}
    \newcommand{\AnnotationTok}[1]{\textcolor[rgb]{0.38,0.63,0.69}{\textbf{\textit{{#1}}}}}
    \newcommand{\CommentVarTok}[1]{\textcolor[rgb]{0.38,0.63,0.69}{\textbf{\textit{{#1}}}}}
    \newcommand{\VariableTok}[1]{\textcolor[rgb]{0.10,0.09,0.49}{{#1}}}
    \newcommand{\ControlFlowTok}[1]{\textcolor[rgb]{0.00,0.44,0.13}{\textbf{{#1}}}}
    \newcommand{\OperatorTok}[1]{\textcolor[rgb]{0.40,0.40,0.40}{{#1}}}
    \newcommand{\BuiltInTok}[1]{{#1}}
    \newcommand{\ExtensionTok}[1]{{#1}}
    \newcommand{\PreprocessorTok}[1]{\textcolor[rgb]{0.74,0.48,0.00}{{#1}}}
    \newcommand{\AttributeTok}[1]{\textcolor[rgb]{0.49,0.56,0.16}{{#1}}}
    \newcommand{\InformationTok}[1]{\textcolor[rgb]{0.38,0.63,0.69}{\textbf{\textit{{#1}}}}}
    \newcommand{\WarningTok}[1]{\textcolor[rgb]{0.38,0.63,0.69}{\textbf{\textit{{#1}}}}}


    % Define a nice break command that doesn't care if a line doesn't already
    % exist.
    \def\br{\hspace*{\fill} \\* }
    % Math Jax compatibility definitions
    \def\gt{>}
    \def\lt{<}
    \let\Oldtex\TeX
    \let\Oldlatex\LaTeX
    \renewcommand{\TeX}{\textrm{\Oldtex}}
    \renewcommand{\LaTeX}{\textrm{\Oldlatex}}
    % Document parameters
    % Document title
    \title{hw7}
    
    
    
    
    
% Pygments definitions
\makeatletter
\def\PY@reset{\let\PY@it=\relax \let\PY@bf=\relax%
    \let\PY@ul=\relax \let\PY@tc=\relax%
    \let\PY@bc=\relax \let\PY@ff=\relax}
\def\PY@tok#1{\csname PY@tok@#1\endcsname}
\def\PY@toks#1+{\ifx\relax#1\empty\else%
    \PY@tok{#1}\expandafter\PY@toks\fi}
\def\PY@do#1{\PY@bc{\PY@tc{\PY@ul{%
    \PY@it{\PY@bf{\PY@ff{#1}}}}}}}
\def\PY#1#2{\PY@reset\PY@toks#1+\relax+\PY@do{#2}}

\@namedef{PY@tok@w}{\def\PY@tc##1{\textcolor[rgb]{0.73,0.73,0.73}{##1}}}
\@namedef{PY@tok@c}{\let\PY@it=\textit\def\PY@tc##1{\textcolor[rgb]{0.24,0.48,0.48}{##1}}}
\@namedef{PY@tok@cp}{\def\PY@tc##1{\textcolor[rgb]{0.61,0.40,0.00}{##1}}}
\@namedef{PY@tok@k}{\let\PY@bf=\textbf\def\PY@tc##1{\textcolor[rgb]{0.00,0.50,0.00}{##1}}}
\@namedef{PY@tok@kp}{\def\PY@tc##1{\textcolor[rgb]{0.00,0.50,0.00}{##1}}}
\@namedef{PY@tok@kt}{\def\PY@tc##1{\textcolor[rgb]{0.69,0.00,0.25}{##1}}}
\@namedef{PY@tok@o}{\def\PY@tc##1{\textcolor[rgb]{0.40,0.40,0.40}{##1}}}
\@namedef{PY@tok@ow}{\let\PY@bf=\textbf\def\PY@tc##1{\textcolor[rgb]{0.67,0.13,1.00}{##1}}}
\@namedef{PY@tok@nb}{\def\PY@tc##1{\textcolor[rgb]{0.00,0.50,0.00}{##1}}}
\@namedef{PY@tok@nf}{\def\PY@tc##1{\textcolor[rgb]{0.00,0.00,1.00}{##1}}}
\@namedef{PY@tok@nc}{\let\PY@bf=\textbf\def\PY@tc##1{\textcolor[rgb]{0.00,0.00,1.00}{##1}}}
\@namedef{PY@tok@nn}{\let\PY@bf=\textbf\def\PY@tc##1{\textcolor[rgb]{0.00,0.00,1.00}{##1}}}
\@namedef{PY@tok@ne}{\let\PY@bf=\textbf\def\PY@tc##1{\textcolor[rgb]{0.80,0.25,0.22}{##1}}}
\@namedef{PY@tok@nv}{\def\PY@tc##1{\textcolor[rgb]{0.10,0.09,0.49}{##1}}}
\@namedef{PY@tok@no}{\def\PY@tc##1{\textcolor[rgb]{0.53,0.00,0.00}{##1}}}
\@namedef{PY@tok@nl}{\def\PY@tc##1{\textcolor[rgb]{0.46,0.46,0.00}{##1}}}
\@namedef{PY@tok@ni}{\let\PY@bf=\textbf\def\PY@tc##1{\textcolor[rgb]{0.44,0.44,0.44}{##1}}}
\@namedef{PY@tok@na}{\def\PY@tc##1{\textcolor[rgb]{0.41,0.47,0.13}{##1}}}
\@namedef{PY@tok@nt}{\let\PY@bf=\textbf\def\PY@tc##1{\textcolor[rgb]{0.00,0.50,0.00}{##1}}}
\@namedef{PY@tok@nd}{\def\PY@tc##1{\textcolor[rgb]{0.67,0.13,1.00}{##1}}}
\@namedef{PY@tok@s}{\def\PY@tc##1{\textcolor[rgb]{0.73,0.13,0.13}{##1}}}
\@namedef{PY@tok@sd}{\let\PY@it=\textit\def\PY@tc##1{\textcolor[rgb]{0.73,0.13,0.13}{##1}}}
\@namedef{PY@tok@si}{\let\PY@bf=\textbf\def\PY@tc##1{\textcolor[rgb]{0.64,0.35,0.47}{##1}}}
\@namedef{PY@tok@se}{\let\PY@bf=\textbf\def\PY@tc##1{\textcolor[rgb]{0.67,0.36,0.12}{##1}}}
\@namedef{PY@tok@sr}{\def\PY@tc##1{\textcolor[rgb]{0.64,0.35,0.47}{##1}}}
\@namedef{PY@tok@ss}{\def\PY@tc##1{\textcolor[rgb]{0.10,0.09,0.49}{##1}}}
\@namedef{PY@tok@sx}{\def\PY@tc##1{\textcolor[rgb]{0.00,0.50,0.00}{##1}}}
\@namedef{PY@tok@m}{\def\PY@tc##1{\textcolor[rgb]{0.40,0.40,0.40}{##1}}}
\@namedef{PY@tok@gh}{\let\PY@bf=\textbf\def\PY@tc##1{\textcolor[rgb]{0.00,0.00,0.50}{##1}}}
\@namedef{PY@tok@gu}{\let\PY@bf=\textbf\def\PY@tc##1{\textcolor[rgb]{0.50,0.00,0.50}{##1}}}
\@namedef{PY@tok@gd}{\def\PY@tc##1{\textcolor[rgb]{0.63,0.00,0.00}{##1}}}
\@namedef{PY@tok@gi}{\def\PY@tc##1{\textcolor[rgb]{0.00,0.52,0.00}{##1}}}
\@namedef{PY@tok@gr}{\def\PY@tc##1{\textcolor[rgb]{0.89,0.00,0.00}{##1}}}
\@namedef{PY@tok@ge}{\let\PY@it=\textit}
\@namedef{PY@tok@gs}{\let\PY@bf=\textbf}
\@namedef{PY@tok@gp}{\let\PY@bf=\textbf\def\PY@tc##1{\textcolor[rgb]{0.00,0.00,0.50}{##1}}}
\@namedef{PY@tok@go}{\def\PY@tc##1{\textcolor[rgb]{0.44,0.44,0.44}{##1}}}
\@namedef{PY@tok@gt}{\def\PY@tc##1{\textcolor[rgb]{0.00,0.27,0.87}{##1}}}
\@namedef{PY@tok@err}{\def\PY@bc##1{{\setlength{\fboxsep}{\string -\fboxrule}\fcolorbox[rgb]{1.00,0.00,0.00}{1,1,1}{\strut ##1}}}}
\@namedef{PY@tok@kc}{\let\PY@bf=\textbf\def\PY@tc##1{\textcolor[rgb]{0.00,0.50,0.00}{##1}}}
\@namedef{PY@tok@kd}{\let\PY@bf=\textbf\def\PY@tc##1{\textcolor[rgb]{0.00,0.50,0.00}{##1}}}
\@namedef{PY@tok@kn}{\let\PY@bf=\textbf\def\PY@tc##1{\textcolor[rgb]{0.00,0.50,0.00}{##1}}}
\@namedef{PY@tok@kr}{\let\PY@bf=\textbf\def\PY@tc##1{\textcolor[rgb]{0.00,0.50,0.00}{##1}}}
\@namedef{PY@tok@bp}{\def\PY@tc##1{\textcolor[rgb]{0.00,0.50,0.00}{##1}}}
\@namedef{PY@tok@fm}{\def\PY@tc##1{\textcolor[rgb]{0.00,0.00,1.00}{##1}}}
\@namedef{PY@tok@vc}{\def\PY@tc##1{\textcolor[rgb]{0.10,0.09,0.49}{##1}}}
\@namedef{PY@tok@vg}{\def\PY@tc##1{\textcolor[rgb]{0.10,0.09,0.49}{##1}}}
\@namedef{PY@tok@vi}{\def\PY@tc##1{\textcolor[rgb]{0.10,0.09,0.49}{##1}}}
\@namedef{PY@tok@vm}{\def\PY@tc##1{\textcolor[rgb]{0.10,0.09,0.49}{##1}}}
\@namedef{PY@tok@sa}{\def\PY@tc##1{\textcolor[rgb]{0.73,0.13,0.13}{##1}}}
\@namedef{PY@tok@sb}{\def\PY@tc##1{\textcolor[rgb]{0.73,0.13,0.13}{##1}}}
\@namedef{PY@tok@sc}{\def\PY@tc##1{\textcolor[rgb]{0.73,0.13,0.13}{##1}}}
\@namedef{PY@tok@dl}{\def\PY@tc##1{\textcolor[rgb]{0.73,0.13,0.13}{##1}}}
\@namedef{PY@tok@s2}{\def\PY@tc##1{\textcolor[rgb]{0.73,0.13,0.13}{##1}}}
\@namedef{PY@tok@sh}{\def\PY@tc##1{\textcolor[rgb]{0.73,0.13,0.13}{##1}}}
\@namedef{PY@tok@s1}{\def\PY@tc##1{\textcolor[rgb]{0.73,0.13,0.13}{##1}}}
\@namedef{PY@tok@mb}{\def\PY@tc##1{\textcolor[rgb]{0.40,0.40,0.40}{##1}}}
\@namedef{PY@tok@mf}{\def\PY@tc##1{\textcolor[rgb]{0.40,0.40,0.40}{##1}}}
\@namedef{PY@tok@mh}{\def\PY@tc##1{\textcolor[rgb]{0.40,0.40,0.40}{##1}}}
\@namedef{PY@tok@mi}{\def\PY@tc##1{\textcolor[rgb]{0.40,0.40,0.40}{##1}}}
\@namedef{PY@tok@il}{\def\PY@tc##1{\textcolor[rgb]{0.40,0.40,0.40}{##1}}}
\@namedef{PY@tok@mo}{\def\PY@tc##1{\textcolor[rgb]{0.40,0.40,0.40}{##1}}}
\@namedef{PY@tok@ch}{\let\PY@it=\textit\def\PY@tc##1{\textcolor[rgb]{0.24,0.48,0.48}{##1}}}
\@namedef{PY@tok@cm}{\let\PY@it=\textit\def\PY@tc##1{\textcolor[rgb]{0.24,0.48,0.48}{##1}}}
\@namedef{PY@tok@cpf}{\let\PY@it=\textit\def\PY@tc##1{\textcolor[rgb]{0.24,0.48,0.48}{##1}}}
\@namedef{PY@tok@c1}{\let\PY@it=\textit\def\PY@tc##1{\textcolor[rgb]{0.24,0.48,0.48}{##1}}}
\@namedef{PY@tok@cs}{\let\PY@it=\textit\def\PY@tc##1{\textcolor[rgb]{0.24,0.48,0.48}{##1}}}

\def\PYZbs{\char`\\}
\def\PYZus{\char`\_}
\def\PYZob{\char`\{}
\def\PYZcb{\char`\}}
\def\PYZca{\char`\^}
\def\PYZam{\char`\&}
\def\PYZlt{\char`\<}
\def\PYZgt{\char`\>}
\def\PYZsh{\char`\#}
\def\PYZpc{\char`\%}
\def\PYZdl{\char`\$}
\def\PYZhy{\char`\-}
\def\PYZsq{\char`\'}
\def\PYZdq{\char`\"}
\def\PYZti{\char`\~}
% for compatibility with earlier versions
\def\PYZat{@}
\def\PYZlb{[}
\def\PYZrb{]}
\makeatother


    % For linebreaks inside Verbatim environment from package fancyvrb.
    \makeatletter
        \newbox\Wrappedcontinuationbox
        \newbox\Wrappedvisiblespacebox
        \newcommand*\Wrappedvisiblespace {\textcolor{red}{\textvisiblespace}}
        \newcommand*\Wrappedcontinuationsymbol {\textcolor{red}{\llap{\tiny$\m@th\hookrightarrow$}}}
        \newcommand*\Wrappedcontinuationindent {3ex }
        \newcommand*\Wrappedafterbreak {\kern\Wrappedcontinuationindent\copy\Wrappedcontinuationbox}
        % Take advantage of the already applied Pygments mark-up to insert
        % potential linebreaks for TeX processing.
        %        {, <, #, %, $, ' and ": go to next line.
        %        _, }, ^, &, >, - and ~: stay at end of broken line.
        % Use of \textquotesingle for straight quote.
        \newcommand*\Wrappedbreaksatspecials {%
            \def\PYGZus{\discretionary{\char`\_}{\Wrappedafterbreak}{\char`\_}}%
            \def\PYGZob{\discretionary{}{\Wrappedafterbreak\char`\{}{\char`\{}}%
            \def\PYGZcb{\discretionary{\char`\}}{\Wrappedafterbreak}{\char`\}}}%
            \def\PYGZca{\discretionary{\char`\^}{\Wrappedafterbreak}{\char`\^}}%
            \def\PYGZam{\discretionary{\char`\&}{\Wrappedafterbreak}{\char`\&}}%
            \def\PYGZlt{\discretionary{}{\Wrappedafterbreak\char`\<}{\char`\<}}%
            \def\PYGZgt{\discretionary{\char`\>}{\Wrappedafterbreak}{\char`\>}}%
            \def\PYGZsh{\discretionary{}{\Wrappedafterbreak\char`\#}{\char`\#}}%
            \def\PYGZpc{\discretionary{}{\Wrappedafterbreak\char`\%}{\char`\%}}%
            \def\PYGZdl{\discretionary{}{\Wrappedafterbreak\char`\$}{\char`\$}}%
            \def\PYGZhy{\discretionary{\char`\-}{\Wrappedafterbreak}{\char`\-}}%
            \def\PYGZsq{\discretionary{}{\Wrappedafterbreak\textquotesingle}{\textquotesingle}}%
            \def\PYGZdq{\discretionary{}{\Wrappedafterbreak\char`\"}{\char`\"}}%
            \def\PYGZti{\discretionary{\char`\~}{\Wrappedafterbreak}{\char`\~}}%
        }
        % Some characters . , ; ? ! / are not pygmentized.
        % This macro makes them "active" and they will insert potential linebreaks
        \newcommand*\Wrappedbreaksatpunct {%
            \lccode`\~`\.\lowercase{\def~}{\discretionary{\hbox{\char`\.}}{\Wrappedafterbreak}{\hbox{\char`\.}}}%
            \lccode`\~`\,\lowercase{\def~}{\discretionary{\hbox{\char`\,}}{\Wrappedafterbreak}{\hbox{\char`\,}}}%
            \lccode`\~`\;\lowercase{\def~}{\discretionary{\hbox{\char`\;}}{\Wrappedafterbreak}{\hbox{\char`\;}}}%
            \lccode`\~`\:\lowercase{\def~}{\discretionary{\hbox{\char`\:}}{\Wrappedafterbreak}{\hbox{\char`\:}}}%
            \lccode`\~`\?\lowercase{\def~}{\discretionary{\hbox{\char`\?}}{\Wrappedafterbreak}{\hbox{\char`\?}}}%
            \lccode`\~`\!\lowercase{\def~}{\discretionary{\hbox{\char`\!}}{\Wrappedafterbreak}{\hbox{\char`\!}}}%
            \lccode`\~`\/\lowercase{\def~}{\discretionary{\hbox{\char`\/}}{\Wrappedafterbreak}{\hbox{\char`\/}}}%
            \catcode`\.\active
            \catcode`\,\active
            \catcode`\;\active
            \catcode`\:\active
            \catcode`\?\active
            \catcode`\!\active
            \catcode`\/\active
            \lccode`\~`\~
        }
    \makeatother

    \let\OriginalVerbatim=\Verbatim
    \makeatletter
    \renewcommand{\Verbatim}[1][1]{%
        %\parskip\z@skip
        \sbox\Wrappedcontinuationbox {\Wrappedcontinuationsymbol}%
        \sbox\Wrappedvisiblespacebox {\FV@SetupFont\Wrappedvisiblespace}%
        \def\FancyVerbFormatLine ##1{\hsize\linewidth
            \vtop{\raggedright\hyphenpenalty\z@\exhyphenpenalty\z@
                \doublehyphendemerits\z@\finalhyphendemerits\z@
                \strut ##1\strut}%
        }%
        % If the linebreak is at a space, the latter will be displayed as visible
        % space at end of first line, and a continuation symbol starts next line.
        % Stretch/shrink are however usually zero for typewriter font.
        \def\FV@Space {%
            \nobreak\hskip\z@ plus\fontdimen3\font minus\fontdimen4\font
            \discretionary{\copy\Wrappedvisiblespacebox}{\Wrappedafterbreak}
            {\kern\fontdimen2\font}%
        }%

        % Allow breaks at special characters using \PYG... macros.
        \Wrappedbreaksatspecials
        % Breaks at punctuation characters . , ; ? ! and / need catcode=\active
        \OriginalVerbatim[#1,codes*=\Wrappedbreaksatpunct]%
    }
    \makeatother

    % Exact colors from NB
    \definecolor{incolor}{HTML}{303F9F}
    \definecolor{outcolor}{HTML}{D84315}
    \definecolor{cellborder}{HTML}{CFCFCF}
    \definecolor{cellbackground}{HTML}{F7F7F7}

    % prompt
    \makeatletter
    \newcommand{\boxspacing}{\kern\kvtcb@left@rule\kern\kvtcb@boxsep}
    \makeatother
    \newcommand{\prompt}[4]{
        {\ttfamily\llap{{\color{#2}[#3]:\hspace{3pt}#4}}\vspace{-\baselineskip}}
    }
    

    
    % Prevent overflowing lines due to hard-to-break entities
    \sloppy
    % Setup hyperref package
    \hypersetup{
      breaklinks=true,  % so long urls are correctly broken across lines
      colorlinks=true,
      urlcolor=urlcolor,
      linkcolor=linkcolor,
      citecolor=citecolor,
      }
    % Slightly bigger margins than the latex defaults
    
    \geometry{verbose,tmargin=1in,bmargin=1in,lmargin=1in,rmargin=1in}
    
    

\begin{document}
    
    \maketitle
    
    

    
    \hypertarget{a.}{%
\subsection{A.)}\label{a.}}

Estimate the attenuation for waves of 30 s period (0.033 Hz) for PcP ray
paths at vertical incidence. Consider only the intrinsic attenuation
along the ray paths; do not include the reflection coefficient at the
core--mantle boundary or geometrical spreading. Be sure to count both
the surface-to- CMB and CMB-to-surface legs. You can get the PREM P
velocities from Figure 1.1 (Copied below) or the values tabulated in
Appendix A. You may estimate the travel times through the different Q
layers in PREM by assuming a fixed velocity within each layer (e.g., use
the average velocity between 3 and 80 km, the average between 80 and 220
km, etc.). Note that in this case, the ``surface'' is assumed to be the
bottom of the ocean at 3 km depth. Compute tstar for PcP. Finally, give the
attenuated amplitude of the PcP ray, assuming the ray had an initial
amplitude of one.

    \begin{tcolorbox}[breakable, size=fbox, boxrule=1pt, pad at break*=1mm,colback=cellbackground, colframe=cellborder]
\prompt{In}{incolor}{1}{\boxspacing}
\begin{Verbatim}[commandchars=\\\{\}]
\PY{c+c1}{\PYZsh{} PcP = Surface through mantle, off core, to surface}

\PY{c+c1}{\PYZsh{} calculate attenuation using eq. 6.86}
\PY{c+c1}{\PYZsh{} A(w) = A\PYZus{}0(w) * exp((\PYZhy{}w*tstar) / 2)}
\PY{c+c1}{\PYZsh{} tstar = int(dt / Q(r)) = 1/Q(r) * int\PYZus{}path(dt)}
\end{Verbatim}
\end{tcolorbox}

    \begin{tcolorbox}[breakable, size=fbox, boxrule=1pt, pad at break*=1mm,colback=cellbackground, colframe=cellborder]
\prompt{In}{incolor}{2}{\boxspacing}
\begin{Verbatim}[commandchars=\\\{\}]
\PY{k+kn}{import} \PY{n+nn}{numpy} \PY{k}{as} \PY{n+nn}{np}

\PY{c+c1}{\PYZsh{} depth of different Q layers}
\PY{n}{depth} \PY{o}{=} \PY{n}{np}\PY{o}{.}\PY{n}{array}\PY{p}{(}\PY{p}{[}
    \PY{p}{(}\PY{l+m+mi}{80}\PY{o}{\PYZhy{}}\PY{l+m+mi}{3}\PY{p}{)}\PY{p}{,}
    \PY{p}{(}\PY{l+m+mi}{220}\PY{o}{\PYZhy{}}\PY{l+m+mi}{80}\PY{p}{)}\PY{p}{,}
    \PY{p}{(}\PY{l+m+mi}{670}\PY{o}{\PYZhy{}}\PY{l+m+mi}{220}\PY{p}{)}\PY{p}{,}
    \PY{p}{(}\PY{l+m+mi}{2891}\PY{o}{\PYZhy{}}\PY{l+m+mi}{670}\PY{p}{)}\PY{p}{,}
    \PY{p}{(}\PY{l+m+mi}{2891}\PY{o}{\PYZhy{}}\PY{l+m+mi}{670}\PY{p}{)}\PY{p}{,}
    \PY{p}{(}\PY{l+m+mi}{670}\PY{o}{\PYZhy{}}\PY{l+m+mi}{220}\PY{p}{)}\PY{p}{,}
    \PY{p}{(}\PY{l+m+mi}{220}\PY{o}{\PYZhy{}}\PY{l+m+mi}{80}\PY{p}{)}\PY{p}{,}
    \PY{p}{(}\PY{l+m+mi}{80}\PY{o}{\PYZhy{}}\PY{l+m+mi}{3}\PY{p}{)}\PY{p}{,}
\PY{p}{]}\PY{p}{)}

\PY{c+c1}{\PYZsh{} calculate mean alpha for each velocity change in Q layer}
\PY{n}{alpha} \PY{o}{=} \PY{n}{np}\PY{o}{.}\PY{n}{array}\PY{p}{(}\PY{p}{[}
    \PY{n}{np}\PY{o}{.}\PY{n}{mean}\PY{p}{(}\PY{p}{(}\PY{l+m+mf}{5.8}\PY{p}{,} \PY{l+m+mf}{6.8}\PY{p}{,} \PY{l+m+mf}{8.11}\PY{p}{,} \PY{l+m+mf}{8.08}\PY{p}{)}\PY{p}{)}\PY{p}{,}
    \PY{n}{np}\PY{o}{.}\PY{n}{mean}\PY{p}{(}\PY{p}{(}\PY{l+m+mf}{8.08}\PY{p}{,} \PY{l+m+mf}{8.02}\PY{p}{,} \PY{l+m+mf}{7.99}\PY{p}{)}\PY{p}{)}\PY{p}{,}
    \PY{n}{np}\PY{o}{.}\PY{n}{mean}\PY{p}{(}\PY{p}{(}\PY{l+m+mf}{8.56}\PY{p}{,} \PY{l+m+mf}{8.66}\PY{p}{,} \PY{l+m+mf}{8.85}\PY{p}{,} \PY{l+m+mf}{8.91}\PY{p}{,} \PY{l+m+mf}{9.13}\PY{p}{,} \PY{l+m+mf}{9.5}\PY{p}{,} \PY{l+m+mf}{10.01}\PY{p}{,} \PY{l+m+mf}{10.16}\PY{p}{,} \PY{l+m+mf}{1.16}\PY{p}{,} \PY{l+m+mf}{10.27}\PY{p}{)}\PY{p}{)}\PY{p}{,}
    \PY{n}{np}\PY{o}{.}\PY{n}{mean}\PY{p}{(}
        \PY{p}{(}\PY{l+m+mf}{10.75}\PY{p}{,} \PY{l+m+mf}{11.07}\PY{p}{,} \PY{l+m+mf}{11.24}\PY{p}{,} \PY{l+m+mf}{11.42}\PY{p}{,} \PY{l+m+mf}{11.58}\PY{p}{,} \PY{l+m+mf}{11.73}\PY{p}{,} \PY{l+m+mf}{11.88}\PY{p}{,} \PY{l+m+mf}{12.02}\PY{p}{,} \PY{l+m+mf}{12.1}\PY{p}{,} \PY{l+m+mf}{12.29}\PY{p}{,} \PY{l+m+mf}{12.42}\PY{p}{,}
        \PY{l+m+mf}{12.54}\PY{p}{,} \PY{l+m+mf}{12.67}\PY{p}{,} \PY{l+m+mf}{12.78}\PY{p}{,} \PY{l+m+mf}{12.9}\PY{p}{,} \PY{l+m+mf}{13.02}\PY{p}{,} \PY{l+m+mf}{13.13}\PY{p}{,} \PY{l+m+mf}{13.25}\PY{p}{,} \PY{l+m+mf}{13.36}\PY{p}{,} \PY{l+m+mf}{13.48}\PY{p}{,} \PY{l+m+mf}{13.60}\PY{p}{,} \PY{l+m+mf}{13.68}\PY{p}{,}
        \PY{l+m+mf}{13.69}\PY{p}{,} \PY{l+m+mf}{13.71}\PY{p}{,} \PY{l+m+mf}{13.72}\PY{p}{)}
    \PY{p}{)}\PY{p}{,}

    \PY{n}{np}\PY{o}{.}\PY{n}{mean}\PY{p}{(}
        \PY{p}{(}\PY{l+m+mf}{10.75}\PY{p}{,} \PY{l+m+mf}{11.07}\PY{p}{,} \PY{l+m+mf}{11.24}\PY{p}{,} \PY{l+m+mf}{11.42}\PY{p}{,} \PY{l+m+mf}{11.58}\PY{p}{,} \PY{l+m+mf}{11.73}\PY{p}{,} \PY{l+m+mf}{11.88}\PY{p}{,} \PY{l+m+mf}{12.02}\PY{p}{,} \PY{l+m+mf}{12.1}\PY{p}{,} \PY{l+m+mf}{12.29}\PY{p}{,} \PY{l+m+mf}{12.42}\PY{p}{,}
        \PY{l+m+mf}{12.54}\PY{p}{,} \PY{l+m+mf}{12.67}\PY{p}{,} \PY{l+m+mf}{12.78}\PY{p}{,} \PY{l+m+mf}{12.9}\PY{p}{,} \PY{l+m+mf}{13.02}\PY{p}{,} \PY{l+m+mf}{13.13}\PY{p}{,} \PY{l+m+mf}{13.25}\PY{p}{,} \PY{l+m+mf}{13.36}\PY{p}{,} \PY{l+m+mf}{13.48}\PY{p}{,} \PY{l+m+mf}{13.60}\PY{p}{,} \PY{l+m+mf}{13.68}\PY{p}{,}
        \PY{l+m+mf}{13.69}\PY{p}{,} \PY{l+m+mf}{13.71}\PY{p}{,} \PY{l+m+mf}{13.72}\PY{p}{)}
    \PY{p}{)}\PY{p}{,}
    \PY{n}{np}\PY{o}{.}\PY{n}{mean}\PY{p}{(}\PY{p}{(}\PY{l+m+mf}{8.56}\PY{p}{,} \PY{l+m+mf}{8.66}\PY{p}{,} \PY{l+m+mf}{8.85}\PY{p}{,} \PY{l+m+mf}{8.91}\PY{p}{,} \PY{l+m+mf}{9.13}\PY{p}{,} \PY{l+m+mf}{9.5}\PY{p}{,} \PY{l+m+mf}{10.01}\PY{p}{,} \PY{l+m+mf}{10.16}\PY{p}{,} \PY{l+m+mf}{1.16}\PY{p}{,} \PY{l+m+mf}{10.27}\PY{p}{)}\PY{p}{)}\PY{p}{,}
    \PY{n}{np}\PY{o}{.}\PY{n}{mean}\PY{p}{(}\PY{p}{(}\PY{l+m+mf}{8.08}\PY{p}{,} \PY{l+m+mf}{8.02}\PY{p}{,} \PY{l+m+mf}{7.99}\PY{p}{)}\PY{p}{)}\PY{p}{,}
    \PY{n}{np}\PY{o}{.}\PY{n}{mean}\PY{p}{(}\PY{p}{(}\PY{l+m+mf}{5.8}\PY{p}{,} \PY{l+m+mf}{6.8}\PY{p}{,} \PY{l+m+mf}{8.11}\PY{p}{,} \PY{l+m+mf}{8.08}\PY{p}{)}\PY{p}{)}\PY{p}{,}

\PY{p}{]}\PY{p}{)}

\PY{c+c1}{\PYZsh{} Q values to outer core from PREM}
\PY{n}{Q} \PY{o}{=} \PY{n}{np}\PY{o}{.}\PY{n}{array}\PY{p}{(}\PY{p}{[}\PY{l+m+mi}{600}\PY{p}{,} \PY{l+m+mi}{80}\PY{p}{,}\PY{l+m+mi}{143}\PY{p}{,} \PY{l+m+mi}{312}\PY{p}{,} \PY{l+m+mi}{312}\PY{p}{,} \PY{l+m+mi}{143}\PY{p}{,} \PY{l+m+mi}{80}\PY{p}{,} \PY{l+m+mi}{600}\PY{p}{]}\PY{p}{)}
\end{Verbatim}
\end{tcolorbox}

    \begin{tcolorbox}[breakable, size=fbox, boxrule=1pt, pad at break*=1mm,colback=cellbackground, colframe=cellborder]
\prompt{In}{incolor}{3}{\boxspacing}
\begin{Verbatim}[commandchars=\\\{\}]
\PY{c+c1}{\PYZsh{} v = d / t \PYZhy{}\PYZgt{} t = d / v}
\PY{n}{dt} \PY{o}{=} \PY{n}{depth} \PY{o}{/} \PY{n}{alpha}
\PY{n}{tstar} \PY{o}{=} \PY{n}{dt} \PY{o}{/} \PY{n}{Q}
\end{Verbatim}
\end{tcolorbox}

    \begin{tcolorbox}[breakable, size=fbox, boxrule=1pt, pad at break*=1mm,colback=cellbackground, colframe=cellborder]
\prompt{In}{incolor}{4}{\boxspacing}
\begin{Verbatim}[commandchars=\\\{\}]
\PY{n}{w} \PY{o}{=} \PY{l+m+mi}{2} \PY{o}{*} \PY{n}{np}\PY{o}{.}\PY{n}{pi} \PY{o}{*} \PY{l+m+mf}{0.033}
\PY{n}{A\PYZus{}0} \PY{o}{=} \PY{l+m+mi}{1}
\PY{n}{A} \PY{o}{=} \PY{p}{[}\PY{p}{]}
\PY{k}{for} \PY{n}{i} \PY{o+ow}{in} \PY{n+nb}{range}\PY{p}{(}\PY{l+m+mi}{0}\PY{p}{,}\PY{l+m+mi}{8}\PY{p}{,}\PY{l+m+mi}{1}\PY{p}{)}\PY{p}{:}
    \PY{n}{A\PYZus{}0} \PY{o}{=} \PY{n}{A\PYZus{}0} \PY{o}{*} \PY{n}{np}\PY{o}{.}\PY{n}{exp}\PY{p}{(}\PY{o}{\PYZhy{}}\PY{n}{w} \PY{o}{*} \PY{n}{tstar}\PY{p}{[}\PY{n}{i}\PY{p}{]} \PY{o}{/} \PY{l+m+mi}{2}\PY{p}{)}
    \PY{n}{A}\PY{o}{.}\PY{n}{append}\PY{p}{(}\PY{n}{A\PYZus{}0}\PY{p}{)}
\PY{n}{att\PYZus{}amp} \PY{o}{=} \PY{n}{A}\PY{p}{[}\PY{o}{\PYZhy{}}\PY{l+m+mi}{1}\PY{p}{]}
\PY{n+nb}{print}\PY{p}{(}\PY{l+s+sa}{f}\PY{l+s+s1}{\PYZsq{}}\PY{l+s+s1}{The final attenuated amplited for the PcP wave is: }\PY{l+s+si}{\PYZob{}}\PY{n}{att\PYZus{}amp}\PY{l+s+si}{\PYZcb{}}\PY{l+s+s1}{\PYZsq{}}\PY{p}{)}
\end{Verbatim}
\end{tcolorbox}

    \begin{Verbatim}[commandchars=\\\{\}]
The final attenuated amplited for the PcP wave is: 0.784300449848276
    \end{Verbatim}

    \hypertarget{b.}{%
\subsection{B.)}\label{b.}}

Using the approximation t\emph{=4t}, compute the attenuated amplitude of
ScS at 30 s period.

    \begin{tcolorbox}[breakable, size=fbox, boxrule=1pt, pad at break*=1mm,colback=cellbackground, colframe=cellborder]
\prompt{In}{incolor}{5}{\boxspacing}
\begin{Verbatim}[commandchars=\\\{\}]
\PY{n}{tstar\PYZus{}b} \PY{o}{=} \PY{l+m+mi}{4} \PY{o}{*} \PY{n}{tstar}
\end{Verbatim}
\end{tcolorbox}

    \begin{tcolorbox}[breakable, size=fbox, boxrule=1pt, pad at break*=1mm,colback=cellbackground, colframe=cellborder]
\prompt{In}{incolor}{6}{\boxspacing}
\begin{Verbatim}[commandchars=\\\{\}]
\PY{n}{w} \PY{o}{=} \PY{p}{(}\PY{l+m+mi}{2} \PY{o}{*} \PY{n}{np}\PY{o}{.}\PY{n}{pi}\PY{p}{)} \PY{o}{/} \PY{l+m+mi}{30}
\PY{n}{A\PYZus{}0} \PY{o}{=} \PY{l+m+mi}{1}
\PY{n}{A} \PY{o}{=} \PY{p}{[}\PY{p}{]}
\PY{k}{for} \PY{n}{i} \PY{o+ow}{in} \PY{n+nb}{range}\PY{p}{(}\PY{l+m+mi}{0}\PY{p}{,}\PY{l+m+mi}{8}\PY{p}{,}\PY{l+m+mi}{1}\PY{p}{)}\PY{p}{:}
    \PY{n}{A\PYZus{}0} \PY{o}{=} \PY{n}{A\PYZus{}0} \PY{o}{*} \PY{n}{np}\PY{o}{.}\PY{n}{exp}\PY{p}{(}\PY{o}{\PYZhy{}}\PY{n}{w} \PY{o}{*} \PY{n}{tstar\PYZus{}b}\PY{p}{[}\PY{n}{i}\PY{p}{]} \PY{o}{/} \PY{l+m+mi}{2}\PY{p}{)}
    \PY{n}{A}\PY{o}{.}\PY{n}{append}\PY{p}{(}\PY{n}{A\PYZus{}0}\PY{p}{)}
\PY{n}{att\PYZus{}amp} \PY{o}{=} \PY{n}{A}\PY{p}{[}\PY{o}{\PYZhy{}}\PY{l+m+mi}{1}\PY{p}{]}
\PY{n+nb}{print}\PY{p}{(}\PY{l+s+sa}{f}\PY{l+s+s1}{\PYZsq{}}\PY{l+s+s1}{The final attenuated amplited for the ScS wave is: }\PY{l+s+si}{\PYZob{}}\PY{n}{att\PYZus{}amp}\PY{l+s+si}{\PYZcb{}}\PY{l+s+s1}{\PYZsq{}}\PY{p}{)}
\end{Verbatim}
\end{tcolorbox}

    \begin{Verbatim}[commandchars=\\\{\}]
The final attenuated amplited for the ScS wave is: 0.37468518513264615
    \end{Verbatim}

    \hypertarget{c.}{%
\subsection{C.)}\label{c.}}

Repeat (a) and (b) for 1 s period (1 Hz) PcP and ScS waves.

    \begin{tcolorbox}[breakable, size=fbox, boxrule=1pt, pad at break*=1mm,colback=cellbackground, colframe=cellborder]
\prompt{In}{incolor}{7}{\boxspacing}
\begin{Verbatim}[commandchars=\\\{\}]
\PY{n}{w} \PY{o}{=} \PY{l+m+mi}{2} \PY{o}{*} \PY{n}{np}\PY{o}{.}\PY{n}{pi} 

\PY{n}{A\PYZus{}0} \PY{o}{=} \PY{l+m+mi}{1}
\PY{n}{A} \PY{o}{=} \PY{p}{[}\PY{p}{]}
\PY{k}{for} \PY{n}{i} \PY{o+ow}{in} \PY{n+nb}{range}\PY{p}{(}\PY{l+m+mi}{0}\PY{p}{,}\PY{l+m+mi}{8}\PY{p}{,}\PY{l+m+mi}{1}\PY{p}{)}\PY{p}{:}
    \PY{n}{A\PYZus{}0} \PY{o}{=} \PY{n}{A\PYZus{}0} \PY{o}{*} \PY{n}{np}\PY{o}{.}\PY{n}{exp}\PY{p}{(}\PY{o}{\PYZhy{}}\PY{n}{w} \PY{o}{*} \PY{n}{tstar}\PY{p}{[}\PY{n}{i}\PY{p}{]} \PY{o}{/} \PY{l+m+mi}{2}\PY{p}{)}
    \PY{n}{A}\PY{o}{.}\PY{n}{append}\PY{p}{(}\PY{n}{A\PYZus{}0}\PY{p}{)}
\PY{n}{att\PYZus{}amp} \PY{o}{=} \PY{n}{A}\PY{p}{[}\PY{o}{\PYZhy{}}\PY{l+m+mi}{1}\PY{p}{]}
\PY{n+nb}{print}\PY{p}{(}\PY{l+s+sa}{f}\PY{l+s+s1}{\PYZsq{}}\PY{l+s+s1}{The final attenuated amplited for the PcP wave is: }\PY{l+s+si}{\PYZob{}}\PY{n}{att\PYZus{}amp}\PY{l+s+si}{\PYZcb{}}\PY{l+s+s1}{\PYZsq{}}\PY{p}{)}

\PY{n}{A\PYZus{}0} \PY{o}{=} \PY{l+m+mi}{1}
\PY{n}{A} \PY{o}{=} \PY{p}{[}\PY{p}{]}
\PY{k}{for} \PY{n}{i} \PY{o+ow}{in} \PY{n+nb}{range}\PY{p}{(}\PY{l+m+mi}{0}\PY{p}{,}\PY{l+m+mi}{8}\PY{p}{,}\PY{l+m+mi}{1}\PY{p}{)}\PY{p}{:}
    \PY{n}{A\PYZus{}0} \PY{o}{=} \PY{n}{A\PYZus{}0} \PY{o}{*} \PY{n}{np}\PY{o}{.}\PY{n}{exp}\PY{p}{(}\PY{o}{\PYZhy{}}\PY{n}{w} \PY{o}{*} \PY{n}{tstar\PYZus{}b}\PY{p}{[}\PY{n}{i}\PY{p}{]} \PY{o}{/} \PY{l+m+mi}{2}\PY{p}{)}
    \PY{n}{A}\PY{o}{.}\PY{n}{append}\PY{p}{(}\PY{n}{A\PYZus{}0}\PY{p}{)}
\PY{n}{att\PYZus{}amp} \PY{o}{=} \PY{n}{A}\PY{p}{[}\PY{o}{\PYZhy{}}\PY{l+m+mi}{1}\PY{p}{]}
\PY{n+nb}{print}\PY{p}{(}\PY{l+s+sa}{f}\PY{l+s+s1}{\PYZsq{}}\PY{l+s+s1}{The final attenuated amplited for the ScS wave is: }\PY{l+s+si}{\PYZob{}}\PY{n}{att\PYZus{}amp}\PY{l+s+si}{\PYZcb{}}\PY{l+s+s1}{\PYZsq{}}\PY{p}{)}
\end{Verbatim}
\end{tcolorbox}

    \begin{Verbatim}[commandchars=\\\{\}]
The final attenuated amplited for the PcP wave is: 0.0006345983116549022
The final attenuated amplited for the ScS wave is: 1.6217938504235532e-13
    \end{Verbatim}

    \hypertarget{d.}{%
\subsection{D.)}\label{d.}}

Repeat your calculations for the PcP and ScS attenuated amplitudes at 1
Hz, but this time, use the Warren and Shearer (2000) Qα values from the
figure below (Figure 6.17). How do the predicted amplitudes compare to
the PREM model predictions? What do these results predict for the
observability of teleseismic S arrivals at 1 Hz?

    \begin{tcolorbox}[breakable, size=fbox, boxrule=1pt, pad at break*=1mm,colback=cellbackground, colframe=cellborder]
\prompt{In}{incolor}{8}{\boxspacing}
\begin{Verbatim}[commandchars=\\\{\}]
\PY{n}{w} \PY{o}{=} \PY{l+m+mi}{2} \PY{o}{*} \PY{n}{np}\PY{o}{.}\PY{n}{pi} 

\PY{n}{Q} \PY{o}{=} \PY{p}{[}\PY{l+m+mi}{1}\PY{o}{/}\PY{l+m+mf}{0.0043}\PY{p}{,}\PY{l+m+mi}{1}\PY{o}{/}\PY{l+m+mf}{0.0007}\PY{p}{,}\PY{l+m+mi}{1}\PY{o}{/}\PY{l+m+mf}{0.0007}\PY{p}{,}\PY{l+m+mi}{1}\PY{o}{/}\PY{l+m+mf}{0.0007}\PY{p}{,}\PY{l+m+mi}{1}\PY{o}{/}\PY{l+m+mf}{0.0007}\PY{p}{,}\PY{l+m+mi}{1}\PY{o}{/}\PY{l+m+mf}{0.0007}\PY{p}{,}\PY{l+m+mi}{1}\PY{o}{/}\PY{l+m+mf}{0.0007}\PY{p}{,}\PY{l+m+mi}{1}\PY{o}{/}\PY{l+m+mf}{0.0043}\PY{p}{]}
\PY{n}{tstar} \PY{o}{=} \PY{n}{dt} \PY{o}{/} \PY{n}{Q}
\PY{n}{tstar\PYZus{}b} \PY{o}{=} \PY{l+m+mi}{4} \PY{o}{*} \PY{n}{tstar}

\PY{n}{A\PYZus{}0} \PY{o}{=} \PY{l+m+mi}{1}
\PY{n}{A} \PY{o}{=} \PY{p}{[}\PY{p}{]}
\PY{k}{for} \PY{n}{i} \PY{o+ow}{in} \PY{n+nb}{range}\PY{p}{(}\PY{l+m+mi}{0}\PY{p}{,}\PY{l+m+mi}{8}\PY{p}{,}\PY{l+m+mi}{1}\PY{p}{)}\PY{p}{:}
    \PY{n}{A\PYZus{}0} \PY{o}{=} \PY{n}{A\PYZus{}0} \PY{o}{*} \PY{n}{np}\PY{o}{.}\PY{n}{exp}\PY{p}{(}\PY{o}{\PYZhy{}}\PY{n}{w} \PY{o}{*} \PY{n}{tstar}\PY{p}{[}\PY{n}{i}\PY{p}{]} \PY{o}{/} \PY{l+m+mi}{2}\PY{p}{)}
    \PY{n}{A}\PY{o}{.}\PY{n}{append}\PY{p}{(}\PY{n}{A\PYZus{}0}\PY{p}{)}
\PY{n}{att\PYZus{}amp} \PY{o}{=} \PY{n}{A}\PY{p}{[}\PY{o}{\PYZhy{}}\PY{l+m+mi}{1}\PY{p}{]}
\PY{n+nb}{print}\PY{p}{(}\PY{l+s+sa}{f}\PY{l+s+s1}{\PYZsq{}}\PY{l+s+s1}{The final attenuated amplited for the PcP is: }\PY{l+s+si}{\PYZob{}}\PY{n}{att\PYZus{}amp}\PY{l+s+si}{\PYZcb{}}\PY{l+s+s1}{\PYZsq{}}\PY{p}{)}

\PY{n}{A\PYZus{}0} \PY{o}{=} \PY{l+m+mi}{1}
\PY{n}{A} \PY{o}{=} \PY{p}{[}\PY{p}{]}
\PY{k}{for} \PY{n}{i} \PY{o+ow}{in} \PY{n+nb}{range}\PY{p}{(}\PY{l+m+mi}{0}\PY{p}{,}\PY{l+m+mi}{8}\PY{p}{,}\PY{l+m+mi}{1}\PY{p}{)}\PY{p}{:}
    \PY{n}{A\PYZus{}0} \PY{o}{=} \PY{n}{A\PYZus{}0} \PY{o}{*} \PY{n}{np}\PY{o}{.}\PY{n}{exp}\PY{p}{(}\PY{o}{\PYZhy{}}\PY{n}{w} \PY{o}{*} \PY{n}{tstar\PYZus{}b}\PY{p}{[}\PY{n}{i}\PY{p}{]} \PY{o}{/} \PY{l+m+mi}{2}\PY{p}{)}
    \PY{n}{A}\PY{o}{.}\PY{n}{append}\PY{p}{(}\PY{n}{A\PYZus{}0}\PY{p}{)}
\PY{n}{att\PYZus{}amp} \PY{o}{=} \PY{n}{A}\PY{p}{[}\PY{o}{\PYZhy{}}\PY{l+m+mi}{1}\PY{p}{]}
\PY{n+nb}{print}\PY{p}{(}\PY{l+s+sa}{f}\PY{l+s+s1}{\PYZsq{}}\PY{l+s+s1}{The final attenuated amplited for the ScS wave is: }\PY{l+s+si}{\PYZob{}}\PY{n}{att\PYZus{}amp}\PY{l+s+si}{\PYZcb{}}\PY{l+s+s1}{\PYZsq{}}\PY{p}{)}
\end{Verbatim}
\end{tcolorbox}

    \begin{Verbatim}[commandchars=\\\{\}]
The final attenuated amplited for the PcP is: 0.25267159680837875
The final attenuated amplited for the ScS wave is: 0.004075920455865417
    \end{Verbatim}

    The values from Warren and Shearer are much higher, and will drastically
increase the observability of teleseismic S arrivals at 1 Hz

    \hypertarget{section}{%
\section{2.}\label{section}}

This question relates to chapter 12.2 in noise seismology. It is
modified from question 2 of chapter 12. Simulate the summations of
cross-correlation functions from uniformly distributed noise sources

    \hypertarget{a.}{%
\subsection{A.)}\label{a.}}

Produce your own version of the 2D example of Figure 12.3c by summing
contributions from cross-correlations of a uniform azimuthal
distribution of plane-wave sources. You should assume the
cross-correlation functions are simple spike functions, as in Figure
12.3b. You will need to choose dθ small enough that you get a smooth
v(t) function, given your chosen sample rate for v(t).

    \begin{tcolorbox}[breakable, size=fbox, boxrule=1pt, pad at break*=1mm,colback=cellbackground, colframe=cellborder]
\prompt{In}{incolor}{9}{\boxspacing}
\begin{Verbatim}[commandchars=\\\{\}]
\PY{k+kn}{from} \PY{n+nn}{scipy} \PY{k+kn}{import} \PY{n}{signal}
\PY{k+kn}{import} \PY{n+nn}{matplotlib}\PY{n+nn}{.}\PY{n+nn}{pyplot} \PY{k}{as} \PY{n+nn}{plt}
\end{Verbatim}
\end{tcolorbox}

    \begin{tcolorbox}[breakable, size=fbox, boxrule=1pt, pad at break*=1mm,colback=cellbackground, colframe=cellborder]
\prompt{In}{incolor}{10}{\boxspacing}
\begin{Verbatim}[commandchars=\\\{\}]
\PY{c+c1}{\PYZsh{} Sample rate}
\PY{n}{fs} \PY{o}{=} \PY{l+m+mi}{10}

\PY{c+c1}{\PYZsh{} window length}
\PY{n}{twin} \PY{o}{=} \PY{l+m+mi}{5}

\PY{c+c1}{\PYZsh{} points}
\PY{n}{t} \PY{o}{=} \PY{n}{np}\PY{o}{.}\PY{n}{linspace}\PY{p}{(}\PY{l+m+mi}{0}\PY{p}{,} \PY{n}{twin}\PY{p}{,} \PY{n+nb}{int}\PY{p}{(}\PY{n}{twin}\PY{o}{*}\PY{n}{fs}\PY{p}{)}\PY{p}{)}


\PY{c+c1}{\PYZsh{} a single Ricker wavelet}
\PY{n}{a} \PY{o}{=} \PY{l+m+mf}{0.1}\PY{c+c1}{\PYZsh{} width of the Ricker wavelet in s}
\PY{c+c1}{\PYZsh{} ta = np.linspace(0,a,int(a*fs)) }
\PY{n}{s} \PY{o}{=} \PY{n}{signal}\PY{o}{.}\PY{n}{ricker}\PY{p}{(}\PY{n+nb}{len}\PY{p}{(}\PY{n}{t}\PY{p}{)}\PY{p}{,} \PY{n}{a}\PY{p}{)}
\PY{n}{plt}\PY{o}{.}\PY{n}{plot}\PY{p}{(}\PY{n}{t}\PY{p}{,}\PY{o}{\PYZhy{}}\PY{n}{s}\PY{p}{)}
\PY{n}{plt}\PY{o}{.}\PY{n}{xlabel}\PY{p}{(}\PY{l+s+s1}{\PYZsq{}}\PY{l+s+s1}{Time in s}\PY{l+s+s1}{\PYZsq{}}\PY{p}{)}
\end{Verbatim}
\end{tcolorbox}

            \begin{tcolorbox}[breakable, size=fbox, boxrule=.5pt, pad at break*=1mm, opacityfill=0]
\prompt{Out}{outcolor}{10}{\boxspacing}
\begin{Verbatim}[commandchars=\\\{\}]
Text(0.5, 0, 'Time in s')
\end{Verbatim}
\end{tcolorbox}
        
    \begin{center}
    \adjustimage{max size={0.9\linewidth}{0.9\paperheight}}{output_16_1.png}
    \end{center}
    { \hspace*{\fill} \\}
    
    \begin{tcolorbox}[breakable, size=fbox, boxrule=1pt, pad at break*=1mm,colback=cellbackground, colframe=cellborder]
\prompt{In}{incolor}{11}{\boxspacing}
\begin{Verbatim}[commandchars=\\\{\}]
\PY{c+c1}{\PYZsh{} initialize the seismograms with zeros}
\PY{n}{n\PYZus{}1\PYZus{}x} \PY{o}{=} \PY{n}{np}\PY{o}{.}\PY{n}{zeros}\PY{p}{(}\PY{n+nb}{len}\PY{p}{(}\PY{n}{t}\PY{p}{)}\PY{p}{)}
\PY{n}{n\PYZus{}1\PYZus{}y} \PY{o}{=} \PY{n}{np}\PY{o}{.}\PY{n}{zeros}\PY{p}{(}\PY{n+nb}{len}\PY{p}{(}\PY{n}{t}\PY{p}{)}\PY{p}{)}

\PY{c+c1}{\PYZsh{} seismic wavespeed}
\PY{n}{c} \PY{o}{=} \PY{l+m+mi}{3}

\PY{c+c1}{\PYZsh{} station location}
\PY{n}{x} \PY{o}{=} \PY{p}{[}\PY{o}{\PYZhy{}}\PY{l+m+mf}{5.}\PY{p}{,}\PY{l+m+mf}{0.}\PY{p}{]}
\PY{n}{y} \PY{o}{=} \PY{p}{[}\PY{l+m+mf}{5.}\PY{p}{,}\PY{l+m+mi}{0}\PY{p}{]}
\PY{n}{r} \PY{o}{=} \PY{l+m+mf}{10.} \PY{c+c1}{\PYZsh{} radius of the circle that has the noise sources}


\PY{c+c1}{\PYZsh{} assume a theta (in radian)}
\PY{n}{theta\PYZus{}1} \PY{o}{=} \PY{n}{np}\PY{o}{.}\PY{n}{pi}\PY{o}{/}\PY{l+m+mi}{3}
\end{Verbatim}
\end{tcolorbox}

    \begin{tcolorbox}[breakable, size=fbox, boxrule=1pt, pad at break*=1mm,colback=cellbackground, colframe=cellborder]
\prompt{In}{incolor}{12}{\boxspacing}
\begin{Verbatim}[commandchars=\\\{\}]
\PY{c+c1}{\PYZsh{} calculate the distance and travel times}
\PY{n}{dt\PYZus{}1\PYZus{}x} \PY{o}{=} \PY{n}{np}\PY{o}{.}\PY{n}{sqrt}\PY{p}{(} 
    \PY{p}{(}\PY{n}{r}\PY{o}{*}\PY{n}{np}\PY{o}{.}\PY{n}{cos}\PY{p}{(}\PY{n}{theta\PYZus{}1}\PY{p}{)} \PY{o}{\PYZhy{}} \PY{n}{x}\PY{p}{[}\PY{l+m+mi}{0}\PY{p}{]}\PY{p}{)}\PY{o}{*}\PY{o}{*}\PY{l+m+mi}{2} \PY{o}{+} \PY{p}{(}\PY{n}{r}\PY{o}{*}\PY{n}{np}\PY{o}{.}\PY{n}{sin}\PY{p}{(}\PY{n}{theta\PYZus{}1}\PY{p}{)} \PY{o}{\PYZhy{}} \PY{n}{x}\PY{p}{[}\PY{l+m+mi}{1}\PY{p}{]}\PY{p}{)}\PY{o}{*}\PY{o}{*}\PY{l+m+mi}{2} 
\PY{p}{)}\PY{o}{/}\PY{n}{c}

\PY{n}{dt\PYZus{}1\PYZus{}y} \PY{o}{=} \PY{n}{np}\PY{o}{.}\PY{n}{sqrt}\PY{p}{(} 
    \PY{p}{(}\PY{n}{r}\PY{o}{*}\PY{n}{np}\PY{o}{.}\PY{n}{cos}\PY{p}{(}\PY{n}{theta\PYZus{}1}\PY{p}{)} \PY{o}{\PYZhy{}} \PY{n}{y}\PY{p}{[}\PY{l+m+mi}{0}\PY{p}{]}\PY{p}{)}\PY{o}{*}\PY{o}{*}\PY{l+m+mi}{2} \PY{o}{+} \PY{p}{(}\PY{n}{r}\PY{o}{*}\PY{n}{np}\PY{o}{.}\PY{n}{sin}\PY{p}{(}\PY{n}{theta\PYZus{}1}\PY{p}{)} \PY{o}{\PYZhy{}} \PY{n}{y}\PY{p}{[}\PY{l+m+mi}{1}\PY{p}{]}\PY{p}{)}\PY{o}{*}\PY{o}{*}\PY{l+m+mi}{2} 
\PY{p}{)}\PY{o}{/}\PY{n}{c}

\PY{n+nb}{print}\PY{p}{(}\PY{l+s+s2}{\PYZdq{}}\PY{l+s+s2}{travel time between sources and receivers }\PY{l+s+s2}{\PYZdq{}}\PY{p}{,}\PY{n}{dt\PYZus{}1\PYZus{}x}\PY{p}{,}\PY{l+s+s2}{\PYZdq{}}\PY{l+s+s2}{ s}\PY{l+s+s2}{\PYZdq{}}\PY{p}{,}\PY{n}{dt\PYZus{}1\PYZus{}y}\PY{p}{,}\PY{l+s+s2}{\PYZdq{}}\PY{l+s+s2}{ s}\PY{l+s+s2}{\PYZdq{}}\PY{p}{)}
\PY{c+c1}{\PYZsh{} indexes where the arrival time will be centered}
\PY{c+c1}{\PYZsh{} shift also with the ricker wavelet width to avoid acausal signals }
\PY{c+c1}{\PYZsh{} (ricker wavelet is centered at a)}
\PY{n}{i\PYZus{}1\PYZus{}x} \PY{o}{=} \PY{n+nb}{int}\PY{p}{(}\PY{n}{dt\PYZus{}1\PYZus{}x}\PY{o}{*}\PY{n}{fs}\PY{p}{)}
\PY{n}{i\PYZus{}1\PYZus{}y} \PY{o}{=} \PY{n+nb}{int}\PY{p}{(}\PY{n}{dt\PYZus{}1\PYZus{}y}\PY{o}{*}\PY{n}{fs}\PY{p}{)}
\PY{n+nb}{print}\PY{p}{(}\PY{n}{i\PYZus{}1\PYZus{}x}\PY{p}{,}\PY{n}{i\PYZus{}1\PYZus{}y}\PY{p}{)}
\end{Verbatim}
\end{tcolorbox}

    \begin{Verbatim}[commandchars=\\\{\}]
travel time between sources and receivers  4.409585518440984  s
2.8867513459481287  s
44 28
    \end{Verbatim}

    \begin{tcolorbox}[breakable, size=fbox, boxrule=1pt, pad at break*=1mm,colback=cellbackground, colframe=cellborder]
\prompt{In}{incolor}{13}{\boxspacing}
\begin{Verbatim}[commandchars=\\\{\}]
\PY{n}{tcorr} \PY{o}{=} \PY{n}{np}\PY{o}{.}\PY{n}{linspace}\PY{p}{(}\PY{o}{\PYZhy{}}\PY{n}{twin}\PY{p}{,}\PY{n}{twin}\PY{p}{,}\PY{n+nb}{int}\PY{p}{(}\PY{l+m+mi}{2}\PY{o}{*}\PY{n+nb}{len}\PY{p}{(}\PY{n}{t}\PY{p}{)}\PY{o}{\PYZhy{}}\PY{l+m+mi}{1}\PY{p}{)}\PY{p}{)}
\end{Verbatim}
\end{tcolorbox}

    \begin{tcolorbox}[breakable, size=fbox, boxrule=1pt, pad at break*=1mm,colback=cellbackground, colframe=cellborder]
\prompt{In}{incolor}{14}{\boxspacing}
\begin{Verbatim}[commandchars=\\\{\}]
\PY{n}{N} \PY{o}{=} \PY{l+m+mi}{1000} \PY{c+c1}{\PYZsh{} we will create N noise sources}
\PY{c+c1}{\PYZsh{} initialize the cross correlations}
\PY{n}{Corr} \PY{o}{=} \PY{n}{np}\PY{o}{.}\PY{n}{zeros}\PY{p}{(}\PY{n}{shape}\PY{o}{=}\PY{p}{(}\PY{n}{N}\PY{p}{,}\PY{n+nb}{len}\PY{p}{(}\PY{n}{tcorr}\PY{p}{)}\PY{p}{)}\PY{p}{)}
\PY{k}{for} \PY{n}{i} \PY{o+ow}{in} \PY{n+nb}{range}\PY{p}{(}\PY{n}{N}\PY{p}{)}\PY{p}{:}
    \PY{n}{theta} \PY{o}{=} \PY{l+m+mi}{2}\PY{o}{*}\PY{n}{np}\PY{o}{.}\PY{n}{pi}\PY{o}{/}\PY{n}{N}\PY{o}{*}\PY{n}{i}
    \PY{c+c1}{\PYZsh{} calculate the distance between each noise source and the receivers}
    \PY{n}{dt\PYZus{}x} \PY{o}{=} \PY{n}{np}\PY{o}{.}\PY{n}{sqrt}\PY{p}{(}
      \PY{p}{(}\PY{n}{r}\PY{o}{*}\PY{n}{np}\PY{o}{.}\PY{n}{cos}\PY{p}{(}\PY{n}{theta}\PY{p}{)} \PY{o}{\PYZhy{}} \PY{n}{x}\PY{p}{[}\PY{l+m+mi}{0}\PY{p}{]}\PY{p}{)}\PY{o}{*}\PY{o}{*}\PY{l+m+mi}{2} \PY{o}{+} \PY{p}{(}\PY{n}{r}\PY{o}{*}\PY{n}{np}\PY{o}{.}\PY{n}{sin}\PY{p}{(}\PY{n}{theta}\PY{p}{)} \PY{o}{\PYZhy{}} \PY{n}{x}\PY{p}{[}\PY{l+m+mi}{1}\PY{p}{]}\PY{p}{)}\PY{o}{*}\PY{o}{*}\PY{l+m+mi}{2}
    \PY{p}{)}\PY{o}{/}\PY{n}{c}

    \PY{n}{dt\PYZus{}y} \PY{o}{=} \PY{n}{np}\PY{o}{.}\PY{n}{sqrt}\PY{p}{(} 
      \PY{p}{(}\PY{n}{r}\PY{o}{*}\PY{n}{np}\PY{o}{.}\PY{n}{cos}\PY{p}{(}\PY{n}{theta}\PY{p}{)} \PY{o}{\PYZhy{}} \PY{n}{y}\PY{p}{[}\PY{l+m+mi}{0}\PY{p}{]}\PY{p}{)}\PY{o}{*}\PY{o}{*}\PY{l+m+mi}{2} \PY{o}{+} \PY{p}{(}\PY{n}{r}\PY{o}{*}\PY{n}{np}\PY{o}{.}\PY{n}{sin}\PY{p}{(}\PY{n}{theta}\PY{p}{)} \PY{o}{\PYZhy{}} \PY{n}{y}\PY{p}{[}\PY{l+m+mi}{1}\PY{p}{]}\PY{p}{)}\PY{o}{*}\PY{o}{*}\PY{l+m+mi}{2} 
    \PY{p}{)}\PY{o}{/}\PY{n}{c}
    
    \PY{n}{i\PYZus{}x} \PY{o}{=} \PY{n+nb}{int}\PY{p}{(}\PY{n}{dt\PYZus{}x}\PY{o}{*}\PY{n}{fs}\PY{p}{)} 
    \PY{n}{i\PYZus{}y} \PY{o}{=} \PY{n+nb}{int}\PY{p}{(}\PY{n}{dt\PYZus{}y}\PY{o}{*}\PY{n}{fs}\PY{p}{)}

    \PY{n}{n\PYZus{}x}\PY{o}{=} \PY{n}{np}\PY{o}{.}\PY{n}{zeros}\PY{p}{(}\PY{n+nb}{len}\PY{p}{(}\PY{n}{t}\PY{p}{)}\PY{p}{)}
    \PY{n}{n\PYZus{}x}\PY{p}{[} \PY{n}{i\PYZus{}x}\PY{p}{:}\PY{n}{i\PYZus{}x}\PY{o}{+}\PY{l+m+mi}{1}\PY{p}{]}\PY{o}{=}\PY{l+m+mi}{1}

    \PY{n}{n\PYZus{}y}\PY{o}{=} \PY{n}{np}\PY{o}{.}\PY{n}{zeros}\PY{p}{(}\PY{n+nb}{len}\PY{p}{(}\PY{n}{t}\PY{p}{)}\PY{p}{)}
    \PY{n}{n\PYZus{}y}\PY{p}{[} \PY{n}{i\PYZus{}y}\PY{p}{:}\PY{n}{i\PYZus{}y}\PY{o}{+}\PY{l+m+mi}{1}\PY{p}{]}\PY{o}{=}\PY{l+m+mi}{1}
    \PY{n}{Corr}\PY{p}{[}\PY{n}{i}\PY{p}{,}\PY{p}{:}\PY{p}{]} \PY{o}{=} \PY{n}{np}\PY{o}{.}\PY{n}{correlate}\PY{p}{(}\PY{n}{n\PYZus{}x}\PY{p}{,}\PY{n}{n\PYZus{}y}\PY{p}{,}\PY{l+s+s1}{\PYZsq{}}\PY{l+s+s1}{full}\PY{l+s+s1}{\PYZsq{}}\PY{p}{)}
    \PY{n}{plt}\PY{o}{.}\PY{n}{plot}\PY{p}{(}
        \PY{n}{tcorr}\PY{p}{,}
        \PY{n}{Corr}\PY{p}{[}\PY{n}{i}\PY{p}{,}\PY{p}{:}\PY{p}{]}\PY{o}{+}\PY{n}{theta}
    \PY{p}{)}
    \PY{n}{plt}\PY{o}{.}\PY{n}{xlim}\PY{p}{(}\PY{p}{[}\PY{o}{\PYZhy{}}\PY{l+m+mi}{5}\PY{p}{,}\PY{l+m+mi}{5}\PY{p}{]}\PY{p}{)}
\PY{n}{plt}\PY{o}{.}\PY{n}{grid}\PY{p}{(}\PY{p}{)}
\end{Verbatim}
\end{tcolorbox}

    \begin{center}
    \adjustimage{max size={0.9\linewidth}{0.9\paperheight}}{output_20_0.png}
    \end{center}
    { \hspace*{\fill} \\}
    
    \begin{tcolorbox}[breakable, size=fbox, boxrule=1pt, pad at break*=1mm,colback=cellbackground, colframe=cellborder]
\prompt{In}{incolor}{15}{\boxspacing}
\begin{Verbatim}[commandchars=\\\{\}]
\PY{n}{plt}\PY{o}{.}\PY{n}{plot}\PY{p}{(}\PY{n}{tcorr}\PY{p}{,}\PY{n}{np}\PY{o}{.}\PY{n}{sum}\PY{p}{(}\PY{n}{Corr}\PY{p}{,}\PY{n}{axis}\PY{o}{=}\PY{l+m+mi}{0}\PY{p}{)}\PY{p}{)}
\PY{n}{plt}\PY{o}{.}\PY{n}{xlim}\PY{p}{(}\PY{o}{\PYZhy{}}\PY{l+m+mi}{5}\PY{p}{,}\PY{l+m+mi}{5}\PY{p}{)}
\PY{n}{plt}\PY{o}{.}\PY{n}{xlabel}\PY{p}{(}\PY{l+s+s1}{\PYZsq{}}\PY{l+s+s1}{Time (s)}\PY{l+s+s1}{\PYZsq{}}\PY{p}{)}
\PY{n}{plt}\PY{o}{.}\PY{n}{ylabel}\PY{p}{(}\PY{l+s+s1}{\PYZsq{}}\PY{l+s+s1}{V(t)}\PY{l+s+s1}{\PYZsq{}}\PY{p}{)}
\PY{n}{plt}\PY{o}{.}\PY{n}{grid}\PY{p}{(}\PY{p}{)}
\end{Verbatim}
\end{tcolorbox}

    \begin{center}
    \adjustimage{max size={0.9\linewidth}{0.9\paperheight}}{output_21_0.png}
    \end{center}
    { \hspace*{\fill} \\}
    
    \hypertarget{b.}{%
\subsection{B.)}\label{b.}}

Show that your numerical integration results are consistent with
equation 12.13 by plotting these analytical functions on the same plots.
The equations in the books are:

    \begin{tcolorbox}[breakable, size=fbox, boxrule=1pt, pad at break*=1mm,colback=cellbackground, colframe=cellborder]
\prompt{In}{incolor}{16}{\boxspacing}
\begin{Verbatim}[commandchars=\\\{\}]
\PY{n}{v\PYZus{}t} \PY{o}{=} \PY{p}{[}\PY{p}{]}
\PY{k}{for} \PY{n}{t} \PY{o+ow}{in} \PY{n}{np}\PY{o}{.}\PY{n}{arange}\PY{p}{(}\PY{o}{\PYZhy{}}\PY{l+m+mi}{5}\PY{p}{,} \PY{l+m+mi}{5}\PY{p}{,} \PY{l+m+mf}{0.1}\PY{p}{)}\PY{p}{:}
    \PY{k}{if} \PY{n+nb}{abs}\PY{p}{(}\PY{n}{t}\PY{p}{)} \PY{o}{\PYZgt{}}\PY{o}{=} \PY{n}{r}\PY{o}{/}\PY{n}{c}\PY{p}{:}
        \PY{n}{v} \PY{o}{=} \PY{l+m+mi}{0}
    \PY{k}{else}\PY{p}{:}
        \PY{n}{v} \PY{o}{=} \PY{l+m+mi}{1} \PY{o}{/} \PY{p}{(}\PY{n}{np}\PY{o}{.}\PY{n}{pi} \PY{o}{*} \PY{n}{np}\PY{o}{.}\PY{n}{sqrt}\PY{p}{(}\PY{p}{(}\PY{n}{r}\PY{o}{*}\PY{o}{*}\PY{l+m+mi}{2} \PY{o}{/} \PY{n}{c}\PY{o}{*}\PY{o}{*}\PY{l+m+mi}{2}\PY{p}{)} \PY{o}{\PYZhy{}} \PY{n}{t}\PY{o}{*}\PY{o}{*}\PY{l+m+mi}{2}\PY{p}{)}\PY{p}{)}
    \PY{n}{v\PYZus{}t}\PY{o}{.}\PY{n}{append}\PY{p}{(}\PY{n}{v}\PY{p}{)}
    \PY{c+c1}{\PYZsh{} break}
\end{Verbatim}
\end{tcolorbox}

    \begin{tcolorbox}[breakable, size=fbox, boxrule=1pt, pad at break*=1mm,colback=cellbackground, colframe=cellborder]
\prompt{In}{incolor}{17}{\boxspacing}
\begin{Verbatim}[commandchars=\\\{\}]
\PY{n}{plt}\PY{o}{.}\PY{n}{plot}\PY{p}{(}
    \PY{n}{np}\PY{o}{.}\PY{n}{arange}\PY{p}{(}\PY{o}{\PYZhy{}}\PY{l+m+mi}{5}\PY{p}{,}\PY{l+m+mi}{5}\PY{p}{,}\PY{l+m+mf}{0.1}\PY{p}{)}\PY{p}{,}
    \PY{n}{v\PYZus{}t}
\PY{p}{)}
\PY{n}{plt}\PY{o}{.}\PY{n}{title}\PY{p}{(}\PY{l+s+s1}{\PYZsq{}}\PY{l+s+s1}{Theoretical Cross\PYZhy{}Correlation}\PY{l+s+s1}{\PYZsq{}}\PY{p}{)}
\PY{n}{plt}\PY{o}{.}\PY{n}{xlabel}\PY{p}{(}\PY{l+s+s1}{\PYZsq{}}\PY{l+s+s1}{Time (s)}\PY{l+s+s1}{\PYZsq{}}\PY{p}{)}
\PY{n}{plt}\PY{o}{.}\PY{n}{ylabel}\PY{p}{(}\PY{l+s+s1}{\PYZsq{}}\PY{l+s+s1}{V(t)}\PY{l+s+s1}{\PYZsq{}}\PY{p}{)}
\PY{n}{plt}\PY{o}{.}\PY{n}{grid}\PY{p}{(}\PY{p}{)}
\end{Verbatim}
\end{tcolorbox}

    \begin{center}
    \adjustimage{max size={0.9\linewidth}{0.9\paperheight}}{output_24_0.png}
    \end{center}
    { \hspace*{\fill} \\}
    
    \hypertarget{c.}{%
\subsection{C.)}\label{c.}}

Perform partial integrations to look at the effects of imperfect noise
source distributions. Only integrate from -pi/4 to pi/4. This assumes
that the sources are only from one side of the station axis. Can you
comment on the cross-correlation, and discuss the possible issues that
will arise in practical examples? Try with angles from pi/4 to 3pi/2,
what do you notice?

    \begin{tcolorbox}[breakable, size=fbox, boxrule=1pt, pad at break*=1mm,colback=cellbackground, colframe=cellborder]
\prompt{In}{incolor}{18}{\boxspacing}
\begin{Verbatim}[commandchars=\\\{\}]
\PY{n}{N} \PY{o}{=} \PY{l+m+mi}{1000} \PY{c+c1}{\PYZsh{} we will create N noise sources}
\PY{c+c1}{\PYZsh{} initialize the cross correlations}
\PY{n}{t} \PY{o}{=} \PY{n}{np}\PY{o}{.}\PY{n}{linspace}\PY{p}{(}\PY{l+m+mi}{0}\PY{p}{,}\PY{n}{twin}\PY{p}{,}\PY{n+nb}{int}\PY{p}{(}\PY{n}{twin}\PY{o}{*}\PY{n}{fs}\PY{p}{)}\PY{p}{)}

\PY{n}{Corr} \PY{o}{=} \PY{n}{np}\PY{o}{.}\PY{n}{zeros}\PY{p}{(}\PY{n}{shape}\PY{o}{=}\PY{p}{(}\PY{n}{N}\PY{p}{,}\PY{n+nb}{len}\PY{p}{(}\PY{n}{tcorr}\PY{p}{)}\PY{p}{)}\PY{p}{)}
\PY{k}{for} \PY{n}{i} \PY{o+ow}{in} \PY{n+nb}{range}\PY{p}{(}\PY{n}{N}\PY{p}{)}\PY{p}{:}
    \PY{n}{theta} \PY{o}{=} \PY{o}{\PYZhy{}}\PY{n}{np}\PY{o}{.}\PY{n}{pi}\PY{o}{/}\PY{p}{(}\PY{l+m+mi}{4}\PY{p}{)}\PY{o}{+}\PY{n}{i}\PY{o}{*}\PY{l+m+mi}{1}\PY{o}{*}\PY{n}{np}\PY{o}{.}\PY{n}{pi}\PY{o}{/}\PY{p}{(}\PY{l+m+mi}{2}\PY{o}{*}\PY{n}{N}\PY{p}{)}
    \PY{c+c1}{\PYZsh{} calculate the distance between each noise source and the receivers}
    \PY{n}{dt\PYZus{}x} \PY{o}{=} \PY{n}{np}\PY{o}{.}\PY{n}{sqrt}\PY{p}{(}
      \PY{p}{(}\PY{n}{r}\PY{o}{*}\PY{n}{np}\PY{o}{.}\PY{n}{cos}\PY{p}{(}\PY{n}{theta}\PY{p}{)} \PY{o}{\PYZhy{}} \PY{n}{x}\PY{p}{[}\PY{l+m+mi}{0}\PY{p}{]}\PY{p}{)}\PY{o}{*}\PY{o}{*}\PY{l+m+mi}{2} \PY{o}{+} \PY{p}{(}\PY{n}{r}\PY{o}{*}\PY{n}{np}\PY{o}{.}\PY{n}{sin}\PY{p}{(}\PY{n}{theta}\PY{p}{)} \PY{o}{\PYZhy{}} \PY{n}{x}\PY{p}{[}\PY{l+m+mi}{1}\PY{p}{]}\PY{p}{)}\PY{o}{*}\PY{o}{*}\PY{l+m+mi}{2}
    \PY{p}{)}\PY{o}{/}\PY{n}{c}

    \PY{n}{dt\PYZus{}y} \PY{o}{=} \PY{n}{np}\PY{o}{.}\PY{n}{sqrt}\PY{p}{(} 
      \PY{p}{(}\PY{n}{r}\PY{o}{*}\PY{n}{np}\PY{o}{.}\PY{n}{cos}\PY{p}{(}\PY{n}{theta}\PY{p}{)} \PY{o}{\PYZhy{}} \PY{n}{y}\PY{p}{[}\PY{l+m+mi}{0}\PY{p}{]}\PY{p}{)}\PY{o}{*}\PY{o}{*}\PY{l+m+mi}{2} \PY{o}{+} \PY{p}{(}\PY{n}{r}\PY{o}{*}\PY{n}{np}\PY{o}{.}\PY{n}{sin}\PY{p}{(}\PY{n}{theta}\PY{p}{)} \PY{o}{\PYZhy{}} \PY{n}{y}\PY{p}{[}\PY{l+m+mi}{1}\PY{p}{]}\PY{p}{)}\PY{o}{*}\PY{o}{*}\PY{l+m+mi}{2} 
    \PY{p}{)}\PY{o}{/}\PY{n}{c}
    
    \PY{n}{i\PYZus{}x} \PY{o}{=} \PY{n+nb}{int}\PY{p}{(}\PY{n}{dt\PYZus{}x}\PY{o}{*}\PY{n}{fs}\PY{p}{)} 
    \PY{n}{i\PYZus{}y} \PY{o}{=} \PY{n+nb}{int}\PY{p}{(}\PY{n}{dt\PYZus{}y}\PY{o}{*}\PY{n}{fs}\PY{p}{)}

    \PY{n}{n\PYZus{}x} \PY{o}{=} \PY{n}{np}\PY{o}{.}\PY{n}{zeros}\PY{p}{(}\PY{n+nb}{len}\PY{p}{(}\PY{n}{t}\PY{p}{)}\PY{p}{)}
    \PY{n}{n\PYZus{}x}\PY{p}{[} \PY{n}{i\PYZus{}x}\PY{p}{:}\PY{n}{i\PYZus{}x}\PY{o}{+}\PY{l+m+mi}{1}\PY{p}{]}\PY{o}{=}\PY{l+m+mi}{1}

    \PY{n}{n\PYZus{}y}\PY{o}{=} \PY{n}{np}\PY{o}{.}\PY{n}{zeros}\PY{p}{(}\PY{n+nb}{len}\PY{p}{(}\PY{n}{t}\PY{p}{)}\PY{p}{)}
    \PY{n}{n\PYZus{}y}\PY{p}{[} \PY{n}{i\PYZus{}y}\PY{p}{:}\PY{n}{i\PYZus{}y}\PY{o}{+}\PY{l+m+mi}{1}\PY{p}{]}\PY{o}{=}\PY{l+m+mi}{1}
    \PY{n}{Corr}\PY{p}{[}\PY{n}{i}\PY{p}{,}\PY{p}{:}\PY{p}{]} \PY{o}{=} \PY{n}{np}\PY{o}{.}\PY{n}{correlate}\PY{p}{(}\PY{n}{n\PYZus{}x}\PY{p}{,}\PY{n}{n\PYZus{}y}\PY{p}{,}\PY{l+s+s1}{\PYZsq{}}\PY{l+s+s1}{full}\PY{l+s+s1}{\PYZsq{}}\PY{p}{)}
\PY{n}{plt}\PY{o}{.}\PY{n}{plot}\PY{p}{(}
    \PY{n}{tcorr}\PY{p}{,}
    \PY{n}{np}\PY{o}{.}\PY{n}{sum}\PY{p}{(}\PY{n}{Corr}\PY{p}{,}\PY{n}{axis}\PY{o}{=}\PY{l+m+mi}{0}\PY{p}{)}\PY{p}{,}
    \PY{l+s+s1}{\PYZsq{}}\PY{l+s+s1}{r\PYZhy{}}\PY{l+s+s1}{\PYZsq{}}\PY{p}{,}
    \PY{n}{label}\PY{o}{=}\PY{l+s+s1}{\PYZsq{}}\PY{l+s+s1}{\PYZhy{}pi/4 to pi/4}\PY{l+s+s1}{\PYZsq{}}\PY{p}{)}

\PY{n}{N} \PY{o}{=} \PY{l+m+mi}{1000}
\PY{k}{for} \PY{n}{i} \PY{o+ow}{in} \PY{n+nb}{range}\PY{p}{(}\PY{n}{N}\PY{p}{)}\PY{p}{:}
    \PY{n}{theta} \PY{o}{=} \PY{o}{+}\PY{l+m+mf}{0.25}\PY{o}{*}\PY{n}{np}\PY{o}{.}\PY{n}{pi}\PY{o}{+}\PY{n}{i}\PY{o}{*}\PY{l+m+mi}{3}\PY{o}{*}\PY{n}{np}\PY{o}{.}\PY{n}{pi}\PY{o}{/}\PY{p}{(}\PY{l+m+mi}{2}\PY{o}{*}\PY{n}{N}\PY{p}{)}
    \PY{c+c1}{\PYZsh{} calculate the distance between each noise source and the receivers}
    \PY{n}{dt\PYZus{}x} \PY{o}{=} \PY{n}{np}\PY{o}{.}\PY{n}{sqrt}\PY{p}{(}
      \PY{p}{(}\PY{n}{r}\PY{o}{*}\PY{n}{np}\PY{o}{.}\PY{n}{cos}\PY{p}{(}\PY{n}{theta}\PY{p}{)} \PY{o}{\PYZhy{}} \PY{n}{x}\PY{p}{[}\PY{l+m+mi}{0}\PY{p}{]}\PY{p}{)}\PY{o}{*}\PY{o}{*}\PY{l+m+mi}{2} \PY{o}{+} \PY{p}{(}\PY{n}{r}\PY{o}{*}\PY{n}{np}\PY{o}{.}\PY{n}{sin}\PY{p}{(}\PY{n}{theta}\PY{p}{)} \PY{o}{\PYZhy{}} \PY{n}{x}\PY{p}{[}\PY{l+m+mi}{1}\PY{p}{]}\PY{p}{)}\PY{o}{*}\PY{o}{*}\PY{l+m+mi}{2}
    \PY{p}{)}\PY{o}{/}\PY{n}{c}

    \PY{n}{dt\PYZus{}y} \PY{o}{=} \PY{n}{np}\PY{o}{.}\PY{n}{sqrt}\PY{p}{(} 
      \PY{p}{(}\PY{n}{r}\PY{o}{*}\PY{n}{np}\PY{o}{.}\PY{n}{cos}\PY{p}{(}\PY{n}{theta}\PY{p}{)} \PY{o}{\PYZhy{}} \PY{n}{y}\PY{p}{[}\PY{l+m+mi}{0}\PY{p}{]}\PY{p}{)}\PY{o}{*}\PY{o}{*}\PY{l+m+mi}{2} \PY{o}{+} \PY{p}{(}\PY{n}{r}\PY{o}{*}\PY{n}{np}\PY{o}{.}\PY{n}{sin}\PY{p}{(}\PY{n}{theta}\PY{p}{)} \PY{o}{\PYZhy{}} \PY{n}{y}\PY{p}{[}\PY{l+m+mi}{1}\PY{p}{]}\PY{p}{)}\PY{o}{*}\PY{o}{*}\PY{l+m+mi}{2} 
    \PY{p}{)}\PY{o}{/}\PY{n}{c}
    
    \PY{n}{i\PYZus{}x} \PY{o}{=} \PY{n+nb}{int}\PY{p}{(}\PY{n}{dt\PYZus{}x}\PY{o}{*}\PY{n}{fs}\PY{p}{)} 
    \PY{n}{i\PYZus{}y} \PY{o}{=} \PY{n+nb}{int}\PY{p}{(}\PY{n}{dt\PYZus{}y}\PY{o}{*}\PY{n}{fs}\PY{p}{)}

    \PY{n}{n\PYZus{}x} \PY{o}{=} \PY{n}{np}\PY{o}{.}\PY{n}{zeros}\PY{p}{(}\PY{n+nb}{len}\PY{p}{(}\PY{n}{t}\PY{p}{)}\PY{p}{)}
    \PY{n}{n\PYZus{}x}\PY{p}{[} \PY{n}{i\PYZus{}x}\PY{p}{:}\PY{n}{i\PYZus{}x}\PY{o}{+}\PY{l+m+mi}{1}\PY{p}{]}\PY{o}{=}\PY{l+m+mi}{1}

    \PY{n}{n\PYZus{}y}\PY{o}{=} \PY{n}{np}\PY{o}{.}\PY{n}{zeros}\PY{p}{(}\PY{n+nb}{len}\PY{p}{(}\PY{n}{t}\PY{p}{)}\PY{p}{)}
    \PY{n}{n\PYZus{}y}\PY{p}{[} \PY{n}{i\PYZus{}y}\PY{p}{:}\PY{n}{i\PYZus{}y}\PY{o}{+}\PY{l+m+mi}{1}\PY{p}{]}\PY{o}{=}\PY{l+m+mi}{1}
    \PY{n}{Corr}\PY{p}{[}\PY{n}{i}\PY{p}{,}\PY{p}{:}\PY{p}{]} \PY{o}{=} \PY{n}{np}\PY{o}{.}\PY{n}{correlate}\PY{p}{(}\PY{n}{n\PYZus{}x}\PY{p}{,}\PY{n}{n\PYZus{}y}\PY{p}{,}\PY{l+s+s1}{\PYZsq{}}\PY{l+s+s1}{full}\PY{l+s+s1}{\PYZsq{}}\PY{p}{)}
\PY{n}{plt}\PY{o}{.}\PY{n}{plot}\PY{p}{(}
    \PY{n}{tcorr}\PY{p}{,}
    \PY{n}{np}\PY{o}{.}\PY{n}{sum}\PY{p}{(}\PY{n}{Corr}\PY{p}{,}\PY{n}{axis}\PY{o}{=}\PY{l+m+mi}{0}\PY{p}{)}\PY{p}{,}
    \PY{l+s+s1}{\PYZsq{}}\PY{l+s+s1}{g\PYZhy{}}\PY{l+s+s1}{\PYZsq{}}\PY{p}{,}
    \PY{n}{label}\PY{o}{=}\PY{l+s+s1}{\PYZsq{}}\PY{l+s+s1}{pi/4 to 3pi/2}\PY{l+s+s1}{\PYZsq{}}
\PY{p}{)}

\PY{n}{plt}\PY{o}{.}\PY{n}{xlim}\PY{p}{(}\PY{p}{[}\PY{o}{\PYZhy{}}\PY{l+m+mi}{5}\PY{p}{,}\PY{l+m+mi}{5}\PY{p}{]}\PY{p}{)}
\PY{n}{plt}\PY{o}{.}\PY{n}{xlabel}\PY{p}{(}\PY{l+s+s1}{\PYZsq{}}\PY{l+s+s1}{Time (s)}\PY{l+s+s1}{\PYZsq{}}\PY{p}{)}
\PY{n}{plt}\PY{o}{.}\PY{n}{ylabel}\PY{p}{(}\PY{l+s+s1}{\PYZsq{}}\PY{l+s+s1}{V(t)}\PY{l+s+s1}{\PYZsq{}}\PY{p}{)}
\PY{n}{plt}\PY{o}{.}\PY{n}{grid}\PY{p}{(}\PY{p}{)}
\PY{n}{plt}\PY{o}{.}\PY{n}{legend}\PY{p}{(}\PY{p}{)}
\end{Verbatim}
\end{tcolorbox}

            \begin{tcolorbox}[breakable, size=fbox, boxrule=.5pt, pad at break*=1mm, opacityfill=0]
\prompt{Out}{outcolor}{18}{\boxspacing}
\begin{Verbatim}[commandchars=\\\{\}]
<matplotlib.legend.Legend at 0x7f971758c340>
\end{Verbatim}
\end{tcolorbox}
        
    \begin{center}
    \adjustimage{max size={0.9\linewidth}{0.9\paperheight}}{output_26_1.png}
    \end{center}
    { \hspace*{\fill} \\}
    
    The correlations seem to be strongly dependant on location relative to
source.

    \begin{tcolorbox}[breakable, size=fbox, boxrule=1pt, pad at break*=1mm,colback=cellbackground, colframe=cellborder]
\prompt{In}{incolor}{ }{\boxspacing}
\begin{Verbatim}[commandchars=\\\{\}]

\end{Verbatim}
\end{tcolorbox}


    % Add a bibliography block to the postdoc
    
    
    
\end{document}
